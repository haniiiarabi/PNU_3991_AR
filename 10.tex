\documentclass[10pt,a4paper]{book}

\begin{document}

\begin{flushleft}
  \!\!\!\!\!\!\!\!\!\!\!\!\!\!\!\textbf{10} \quad CHAPTER ONE
\end{flushleft}


\!\!\!\!\!\!\!\!\!use.In addition,it can be assumed that these skills will change and cvolve as the net work and its tools evolve .

As Figure 1.1 also illustrates,our listing of Internet skills begins with "Internet efficacy." Efficacy has long been associated with competence and accomplishment.Internet efficacy means the confidence and willingness to learn to use new tools and to become competent at applying these tools to authentic problems.Bandura's (1977) pioneering work illustrates the importance of having the confidence to learn new skills and acquire new attitudes.A major source in creating the digital divide (those with technological skills and access to network technologies and those without) lies beyond economic and class attributes,and includes the self-confidence necessary to attempt to use new networking tools.The effective e-researcher has to have the confidence and willingness to experiment with and learn to use network tools---even those that are unfamiliar and,at least initially,are perceived as highly complex and perhaps intimidating.Eastin and LaRose (2000) developed an eight-item scale to reliably assess(and in a sense behaviorally define a measurement for) Internet self-efficacy.This scale includes items such as "I feel comfortable describing network technology,software,searching for information and learning Internet applications."They determined that it was not Internet skill alone that determined competency,but rather a strong sense of Internet efficacy that allowed users to effectively adapt to the requirements of working in this transitory environment.

Mental models of how the Internet works and the way that variouse organizations and resources function and communicate on the network are also needed to be an effective e-researcher.Mental models are usually graphical but can be abstract renderings or pictures of the way in which ideas,processes,or practices arw organized and related to each other.They help us to predict,anticipate,and manipulate artifacts and structures around us.Construction of accurate mental modelse of the Internet and the way in which practitioners use it is necessary for the development of Internet skills and competencies.For example,it is necessary to have access to a mental model of asynchronous decision making and robotic control before one can even imagine how a group of people could plant and harvest a "virtual garden" as was done in the TeleGarden project in 1996(see http://www.use.edu/dept/garden/).Readersinterested in exploring Internet mental models will probably enjoy Mark Stefik's 1996 book,\emph{\textit{\textmd{Internet Dreams:Archerypes,Mytbs,and Metaphors.}}} 

Access is perhaps the most obvious prerequisite of skillfull Internet use. However,access is more complicated than what might be initially apparenet.There are different types of access,some related to the speed with which Internet resources can be used and others related to the capacity of the computer to render and disply complex graphic images.Our own experiences convince us that the speed with which images appear is directly related to both satisfaction and persistence.In addition,use of many multimedia resources,especially those requiring extensive video,are restricted to those with both high bandwidth connection and equipment new and powerfull enough to render the images in real time.Access is also available to researchers differentially in regard to both time and space.Those whose access to the Internet is restricted to home or office,or whose other personal commitments do not allow evening or weekend access,will likely not be as effective in their e-research projects as those whose access 

\end{document}
