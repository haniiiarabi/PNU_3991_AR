\documentclass[10pt,a4paper]{book}

\begin{document}

\begin{flushright}
  INTRODUCTION \quad \textbf{11}
\end{flushright}


\!\!\!\!\!\!\!\!\!is available "anytime/anywhere." Access is also restricted for those who have a variety of physical or mental disabilities.Much necessary e-research is needed to examine and develop prosthetics that allow all people to access and benefit from the Net.


Mastery of appropriate terminology is important in any field and especially so when the field is expanding and new terms are routinely introduced.There are many useful terminology reference sites and glossaries that can be likes to as needed (see especially Jenkins,2000).However,a new e-researcher may fins that reviewing online tutorials such as those found at the Web Teacher site at http://www.webteacher.org/or reading through a generic introductory text such as Wing,Whitehead,and Maran's \textmd{\textit{nternet and World Wide Web simplified(1999)}} can be most useful in acquiring basic terminology and functionality detail.

Eastion and Larose(2000)found that " prior Internet experience was the strongest predictor of Internet self-efficacy. Up to two years' experience may be required to achieve sufficient self-efficacy." While we would agree that time on task is an important component of Internet self-sfficacy and effective use of the Internet,we do not think that tow years of use is a prerequisite to doing effective e-research.However,we all know colleagues,friends,and family members who acquire Internet skills and related Internet self-efficacy at vastly different speeds.Some ofuse,for example,seem to have a learning style or motivation pereference that makes us predisposed to enjoy the challenges assoeiated with the types of learning experiences founs on the networks.Alternatively,others of us find learning these skills to be an srduous task.Our advice to the beginning e-reaearcher is to expect to spend time,some of which will involve exploration down blind alleys,The investment in time most often results in serendipitous returns---much of which will have application in later networking tasks.

Network skills come from systematic efforts that reflect the use of the technology.some of use find that our skills are improved rapidly through enrollment in face-to-face or online courses.Others of us develop our skills through self-study and even random exploration.Whatever way you learn best,you can be assured that,over time,you will see a sharp increase in your networking and research skill as you undertake and complete e-research projects.

Finally,troubleshooting on the Internet is a particulary useful skill as many Net technologies are not fully developed or optimized for inexperienced users.As your self-efficacy as an e-researcher increases you will find both the opportunity and the need to develop trouble-shooting skills related to both Internet hardware and software.Our only advice is to use your research skill in these practical applications, searching, studying,and finding assistance from a variety of online and localized resource.It follows that you can be helpful and acquire a great deal of knowledge by collaborating with and assisting others struggling with networked tools.

When you expand your Internet skills and Internet self-efficacy,and combine it with your existing knowledge and skills as a researcher,you will be able to do effective e-research.We hope this text serves as a source of information that will turn into knowledge as you apply the information to personally relevant contexts.This knowledge bridges the tools and culture of the network and the established tradition of academic or commercial research.Increasing your e-research skill will increase your capacity to apply them in novel and effective ways.\small There are far too many perplexing

\end{document}
