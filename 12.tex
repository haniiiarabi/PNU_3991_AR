\documentclass[10pt,a4paper]{book}

\begin{document}

\begin{flushleft}
  \textbf{\!\!\!\!\!\!\!\!\!\!\!\!12} \qquad CHAPTER ONE
\end{flushleft}

\!\!\!\!\!\!\!\!\!questions and important,unsolved problems crying for solutions in both the networked and non-networked worlds of education.We hope this text helps us in at least a small way to contribute to their solution.

\begin{flushleft}
  \textbf{\!\!\!\!\!\!\!\!\!\!\!\!PERILS OF e-RESEARCH}
\end{flushleft}

\!\!\!\!\!\!\!\!\!\!we are not immune to the occasionally well-argued and always vitriolic critics of new technology applications in research and education (for example,Neil Postman and David Noble).However,most of use who use the Net do not hold an extreme view of technological determinism. Rather, our moderate view of technological determinism acknowledges how we both use,and are used by , technology. The Net , like all technology,is capable of amplifying and extending the best and the worst of human nature---including that enacted in the practice of educational research.Indeed,it is easier now to conduct shoddy research,to more quickly and broadly disseminate incorrect or misleading results,to more readily exploit the trust of research subjeects,to more easily plagisrize the work of others,and to rely on breadth rather than depth of coverage on most any scholarly topic.

As in other times of rapid change,the ethical,moral,and legal checks to such behavior often lag behind the capacity if the unscrupulous or uncaring to profit from their application.Therefore e-researchers are cautioned always to reflect before acting,to seek the counsel of others within and outside their own community,and to be scrupulously honest and open when explaining and documenting their activities.Doubtlessly,we will make mistakes as we pursue new knowledge using new techniques.However,this is no reason to abandon our effort.Progress in both humanity and in scientifix research has always advanced unevenly,making false starts and reaching dead ends.E-research offers no panacea---it guarantees neither efficacy nor quality of result.Yet,we are convinced that the promise exceeds the peril.We write this book with the hope that researchers will adopt e-research techniques and tools with the ever-present critical edge that defines all effective research.

\begin{flushleft}
  \textbf{\!\!\!\!\!\!\!\!\!\!\!\!ON NAMING THE NET}
\end{flushleft}

\!\!\!\!\!\!\!\!\!\!Throughout the creation of this text we have struggled with the task of consistently naming the natwork environment that is the context in which our research takes palsce.Often we use the term \textit{online} to refer to networked behavior,however we are becoming increasingly aware that wireless networking removes even the line from online activity $!$ The more technical term Internet may be more accurate,but as networking progresses beyond the original Transmission Control Protocol/Internet Protocol(TCP/TP)set of standards that defined the original Internet,this word too seems too confining.We rigorously reject the term \emph{virtuall} with its connotations of unreality,since we are convinced that online behavior and the interactions and feelings that it invokes are very real to must participands.For similar reasons we do not use the term \emph{cyberspace},because we believe it belongs more appropriately within the slightly

\end{document}
